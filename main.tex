\documentclass[12pt, oneside]{ctexbook}
%%务必用XeLaTeX,否则封面页信息会不对齐!
%宏包另有需求自行添加!

\usepackage[left=2.5cm,right=2.5cm,top=2.5cm,bottom=2.5cm]{geometry}
\usepackage{ctex}
\usepackage{titlesec}
% \usepackage{titletoc}
\usepackage{appendix}
%\usepackage[colorlinks]{hyperref}
\usepackage{hyperref}
\usepackage{graphicx}
\usepackage{subfigure}
\usepackage{amsmath,amsfonts,amssymb,amsthm}
\usepackage{booktabs}
\usepackage{listings,xcolor}
\usepackage{setspace}
\usepackage{fancyhdr}
\usepackage{appendix}
\usepackage{cite}
\usepackage{indentfirst}
\usepackage{pdfpages}
\usepackage{float}  % 防止图片乱跑

% 首行缩进两个10pt大小的中文字符
\setlength{\parindent}{2em}

% 设置章节前空行
\titlespacing{\chapter}{0pt}{-20pt}{1\baselineskip}
\setlength{\headheight}{14.5pt} % 页眉高度
% 公式树代码支持
\usepackage{tikz}
\usetikzlibrary{chains,scopes,positioning,backgrounds,shapes,fit,shadows,calc,arrows.meta}
\RequirePackage{amsmath,mathrsfs,amsfonts,amssymb}
\tikzset{
	head/.style = {fill = orange!90!blue},
	tail/.style = {fill = blue!70!yellow},
	fun/.style = {fill = orange!90!blue},
	num/.style = {fill = blue!70!yellow},
	gry/.style = {fill = gray}
}
\usetikzlibrary{patterns,backgrounds}

%%下面两个命令显示PDF文件行号,方便修改!
% \usepackage{lineno}
% \linenumbers

%题目
\newcommand{\cntitle}{\textbf{论文标题}}
\newcommand{\entitle}{\textbf{Title of the Thesis}}

% 定义封面信息
% 学生姓名,指导老师姓名和指导老师职称默认为3个字,如果为2个字,中间用\quad隔开.要是四个字自行修改!
\newcommand{\class}{19专业0S}
\newcommand{\studentnumber}{2019000000}
\newcommand{\name}{姓\quad 名}
\newcommand{\adviser}{导\quad 师}
\newcommand{\professionaltitle}{教\quad 授}

%定义时间
\newcommand{\cnyear}{二〇二三年六月}
\newcommand{\enyear}{2023}

%定义中英文摘要环境
\newenvironment{enabstract}{%
  \par%\small
  \noindent\mbox{}\hfill\hfill\mbox{}\par
  \vskip 2.5ex}{\par\vskip 2.5ex}
\newenvironment{cnabstract}{%
  \par%\small
  \noindent\mbox{}\hfill\hfill\mbox{}\par
  \vskip 2.5ex}{\par\vskip 2.5ex}

%定义上标引用
\newcommand{\upcite}[1]{\textsuperscript{\textsuperscript{\cite{#1}}}}

%定义引用格式
\newcommand{\myeqref}[1]{\eqref{#1}式}
\newcommand{\figref}[1]{图\ref{#1}}
\newcommand{\tabref}[1]{表\ref{#1}}

%定义公式等标号
\numberwithin{equation}{section}
\numberwithin{table}{section}
\numberwithin{figure}{section}

%定义定理(如有需要自行添加)
\newtheorem{lemma}{引理}[section]
\newtheorem{theorem}{定理}[section]
\newtheorem{example}{例}[section]
\newtheorem{remark}{注}[section]
\newtheorem{corollary}{推论}[section]
\newtheorem{definition}{定义}[section]

%\def\proof{\noindent{\heiti 证明 }}


%定义代码环境
% 设置matlab代码环境
\lstnewenvironment{matlab}[1][]{
	\lstset{
		language=Matlab,
		% deletekeywords={disp},% 可在此行去除特定关键词的语法高亮	
        % morekeywords={ r , K , alpha , beta , gamma , mu , D1 , D3 , d }	
		keywordstyle=\color{blue},% 设置关键词样式
		stringstyle=\color[rgb]{0.75,0,0.75},% 设置字符串样式
		commentstyle=\songti\color[rgb]{0,0.5,0},% 设置注释样式
		#1
	}
}{}

% 设置python代码环境
\lstnewenvironment{python}[1][]{
	\lstset{
		language=Python,
		keywordstyle=\color[RGB]{255,119,0},
		morekeywords={as},
		deletekeywords={print},
		keywordstyle=[2]\color[RGB]{144,0,144},
		stringstyle=\color[RGB]{0,170,0},
		commentstyle=\songti\color[RGB]{221,0,0},
		basicstyle=\tiny\ttfamily,
		lineskip=-1pt, % 将行距缩小1pt
		breaklines=true, % 允许自动断行
  		breakindent=2em, % 设置自动断行后新行的缩进量为2个em
		#1
	}
}{}

%标题设置
\titleformat{\chapter}{\heiti\centering\Large}{\heiti{第\chinese{chapter}章}}{1em}{\heiti}
\titleformat{\section}{\large}{\textbf{\thesection}}{1em}{\heiti}
\titleformat{\subsection}{\normalsize}{\textbf{\thesubsection}}{1em}{\heiti}

%页眉设置
\pagestyle{fancy}
\lhead{}
\chead{\heiti 江苏理工学院数理学院\enyear 届毕业论文}
\rhead{}

\begin{document}
%logo必须放在同一个文件夹下
%封面页
\thispagestyle{empty}
\begin{figure}[t]
	\parbox[b]{2cm}{
		\includegraphics[width=\textwidth]{logo}
	}
\end{figure}

\begin{center}
	\quad \\
	\quad \\
	\heiti \zihao{1} 本\ 科\ 毕\ 业\ 论\ 文
	\vskip 3cm
	\text{\zihao{2} \heiti 论文标题}

\end{center}
	\vskip 2cm


\begin{center}
	\songti 
	\fontsize{14}{13}\selectfont 
	学院名称:
	\underline{~~~~~~~~~~~~~~~~~凌波学院~~~~~~~~~~~~~~~~~~~~}
		
	专\qquad 业:
	\underline{~~~~~~论文排版与编辑工程(师范)~~~~~~~}
		
	班\qquad 级:
	\underline{~~~~~~~~~~~~~~~~\class~~~~~~~~~~~~~~~~~~~}
		
	学\qquad 号:
	\underline{~~~~~~~~~~~~~~~\studentnumber~~~~~~~~~~~~~~~~~~~}
		
	姓\qquad 名:
	\underline{~~~~~~~~~~~~~~~~~~~\name~~~~~~~~~~~~~~~~~~~~~}
		
	指导老师姓名:
	\underline{~~~~~~~~~~~~~\adviser~~~~~~~~~~~~~~~~~~~~~}
		
	指导老师职称:
	\underline{~~~~~~~~~~~~~\professionaltitle~~~~~~~~~~~~~~~~~~~~~}
		
	\vskip 1.5cm
		
	\cnyear
\end{center} 
	
%版权页
\thispagestyle{empty}
\begin{center}
\text{ \heiti \Large 江苏理工学院本科毕业设计(论文)原创性声明}


~\\
\end{center}

本人郑重声明所呈交的毕业设计(论文),是本人在导师的指导下进行研究工作所取得的成果.尽我所知,除了文中特别加以标注和引用的内容外,本论文中不含任何其他个人或集体已经发表或撰写过的研究成果.对本文的研究做出重要贡献的个人和集体,均已在文中作了明确的说明或标注.

\vspace{40pt}
论文作者签名: \underline{~~~~~~~~~~~~~~~~~~}
\text{~~~~~~~~~~~~~~~~~~~~~}
日期: \underline{~2023~}  年
\underline{~6~}  月
\underline{~1~} 日


\vspace{80pt}

\begin{center}
\text{\heiti \Large 江苏理工学院本科毕业设计(论文)版权使用授权声明}
	
~\\
\end{center}


本人完全了解江苏理工学院关于本科生毕业设计(论文)工作规程之规定,即本校 学生毕业设计(论文)成果所有权归学校.江苏理工学院有权保留并向有关部门或机 构送交学位论文的复印件和电子文档,可以将本学位论文的全部或部分内容提交至各 类数据库进行发布和检索,也可以采用影印、缩印、扫描等方式保存和发布文本,并 将文本提供给读者查阅、借阅和参考.
\vspace{40pt}


论文作者签名: \underline{~~~~~~~~~~~~~~~~~~}
\text{~~~~~~~~~~~~~~~~~~~~~}
导师签名:\underline{~~~~~~~~~~~~~~~~~~}


日期: \underline{~2023~}  年
\underline{~6~}  月
\underline{~1~} 日
\text{~~~~~~~~~~~~~~~~~~~~~~}
日期: \underline{~2023~}  年
\underline{~6~}  月
\underline{~1~} 日
	
%以上为封面页,版权页(无需修改)
\setcounter{page}{1}
\pagenumbering{Roman}
\newpage
\begin{center}
    \Large\heiti \cntitle
\end{center}
\begin{cnabstract}
    \addcontentsline{toc}{chapter}{摘要}
    {\large\heiti 摘要:} 遗传规划算法是遗传算法的一个分支,是一种基于遗传算法原理的启发式优化算法.

第一章简述了.

第二章,我们.

第三章,我们.

第四章对全文的工作进行了总结,并对未来的研究方向进行了展望.

{\large\heiti 关键词:} 时间序列; 遗传规划

\end{cnabstract}

\newpage
\begin{center}
	\large{\entitle}
\end{center}

\begin{enabstract}
    \addcontentsline{toc}{chapter}{Abstract}
    {\small
    \textbf{Abstract: }{\footnotesize In this paper, we.}

{\footnotesize In Chapter 1, we.}

{\footnotesize In Chapter 2, we.}

{\footnotesize In Chapter 3, we.}

{\footnotesize In Chapter 4, we summarize the work of the entire paper and prospects for future research directions.}

\textbf{Keywords: }{\footnotesize time series, genetic programming}
    }
\end{enabstract}
%目录页
\newpage
%\begin{spacing}{0.5}
\setcounter{tocdepth}{2}
\tableofcontents
%\end{spacing}

%正文
\newpage
\setcounter{page}{2}
\pagenumbering{arabic}

%%以下开始为正文部分,推荐使用\include{file}命令输入各章节

\include{inputchap1}

%参考文献
\addcontentsline{toc}{chapter}{参考文献}
\bibliographystyle{gbt7714-2005}
\begin{spacing}{1}
\bibliography{ref}
\end{spacing}

\titleformat{\chapter}{\centering\Large\bfseries}{附录\thechapter}{1em}{}
\begin{appendix}

\chapter{Python源程序}



\begin{spacing}{0.8}
\chapter{本科期间的论文成果、科研项目和获奖情况}
\section{论文成果}
\begin{enumerate}
  \item
  本科期间发表的论文(参考文献格式)
\end{enumerate}


% \newpage
\section{科研项目}
\begin{enumerate}
  \item 江苏省大学生创新创业训练计划, 省级重点项目, 项目代码, 项目名称, 2099-05 至 2100-05, 999 万, 已结题, 主持.
\end{enumerate}

\section{获奖情况}
插入奖状等扫描件
\end{spacing}


\chapter*{致谢}
\addcontentsline{toc}{chapter}{致谢}

青葱四年,落笔为终。

\begin{flushright}
\name

2023年5月
\end{flushright} 
\end{appendix}



\end{document} 